\chapter{Conclusion and Outlook}
\label{chap:08_conclusion-outlook}

This chapter presents the conclusion of this thesis. Additionally, an outlook is given for further research based on the results of \Chap{chap:07_evaluation}.


% ===========================================
% ===========================================
\section{Conclusion}
\label{sec:08_conclusion}
% Intro
In this thesis, a scalable computing environment for training machine learning models using Apache Spark was implemented. In addition, the training process is automated using an automated deployment pipeline.
% The research questions
In \Chap{chap:01_introduction}, three research questions for this thesis are defined.


% RQ1
The first research question is about implementing a solution to automatically scale the number of workers in an Apache Spark cluster.
% The solution
For this research question, a computing environment is implemented according to the autonomic computing architecture.
% Autonomic computing
It consists of an autonomic manager and an Apache Spark cluster.
% Docker
For simplicity, all components of the computing environment were deployed with Docker.
% manager
The autonomic manager is implemented according to the \hyperlink{abbr:mape}{MAPE} architecture and responsible for monitoring and automatically scaling the Apache Spark workers.
% monitoring system
For the implementation of the autonomic manager, Prometheus, cAdvisor, and DCGM-Exporter are used to create a monitoring system.
% Auto-Scaler
In addition, an \textit{Auto-Scaler} is implemented using Python 3 that automatically fetches performance metrics from the monitoring system and scales the Apache Spark worker nodes.
% personal
The decision to use Docker for containerization contributed to the simplicity of the implementation. Additionally, using Docker Swarm significantly simplified the scaling process of worker nodes. The \textit{Auto-Scaler} is able to fetch meaningful performance metrics from the monitoring system and make scaling decisions in real-time.


% RQ2
The second research question is about how to accelerate the Apache Spark cluster's computational power using \hyperlink{abbr:gpu}{GPUs}.
% RAPIDS
This research question is implemented by extending the existing Apache Spark cluster with the NVIDIA RAPIDS plugin.
% personal
Using the RAPIDS plugin to accelerate the Apache Spark cluster, achieved the expected results. Furthermore, the installation process is very intuitive to enable Apache Spark to leverage \hyperlink{abbr:gpu}{GPUs}. However, the RAPIDS plugin lacks in available features. Firstly, only \hyperlink{abbr:xgboost}{XGBoost} is \hyperlink{abbr:gpu}{GPU} accelerated and it cannot accelerate Apache Spark's ML library. Second, RAPIDS allocates \hyperlink{abbr:gpu}{GPUs} exclusively for an executor which caused problems during the implementation on a shared host.


% RQ3
The final research question is about how to automate the training of a machine learning model.
%
To accommodate this research question, a \hyperlink{abbr:ci}{CI} pipeline is implemented using GitLab CI/CD. The \hyperlink{abbr:ci}{CI} pipeline triggers when a change is committed to the application source code.
% How
It consists of a \textit{train-stage} that automatically deploys a \textit{spark-submit} Docker container to the Apache Spark cluster.
% spark-submit
The \textit{spark-submit} container executes the spark-submit executable to perform the application on the Apache Spark cluster.
% Gitlab runner
To deploy a \textit{spark-submit} container from the \textit{train-stage} inside the Apache Spark cluster, a GitLab Runner is created, which runs on the same host.
% personal
With GitLab CI/CD, the automation of the Machine Learning model training using a \hyperlink{abbr:ci}{CI} pipeline is a straightforward process. However, it requires to implement the \hyperlink{abbr:ci}{CI} pipeline separately for each project. Therefore, developers will probably end up coping and pasting boilerplate code to setup the pipeline, which can be further simplified.


% Results
The implementation is evaluated with a variety of experiments using a classification and regression machine learning algorithm.
% GPU
The \hyperlink{abbr:gpu}{GPU}-accelerated Apache Spark cluster significantly outperformed a \hyperlink{abbr:cpu}{CPU}-only cluster.
% AUto-Scaler
However, the \textit{Auto-Scaler} implementation is not able to improve the performance of the Apache Spark cluster. Furthermore, it increases the overall execution time. In order to scale \hyperlink{abbr:gpu}{GPU}-accelerated worker nodes, the \textit{Auto-Scaler} implementation requires further work. Instead of horizontal-scaling, the \textit{Auto-Scaler} can be extended using different approaches to automatically scale the environment's computational power.


% personal opinion
Overall, I can recommend to use the implementation of this thesis, except die Auto-Scaler.
Overall, the thesis implementation can be recommended to use in future projects.
%
It is recommended to 

This thesis achieved to answer the given research questions. The horizontal scaling approach was not the right choice to scale the cluster. In future works a different scaling approach should be considered. The choice of technologies successfully achieved to create a meaningful outcome.Future works should consider to 


% ===========================================
% ===========================================
\section{Outlook}
% Intro
Mentioned in \Sec{sec:08_conclusion}, the \textit{Auto-Scaler} implementation was not able to improve the performance of training machine learning models on the Apache Spark cluster.
% A possible reason
A possible reason for the execution time increase is that Apache Spark is not able to efficiently distribute the workload of an application while workers are added to the cluster during the execution of an application.
% CI pipeline
Additionally, mentioned in \Sec{sec:08_conclusion} as well, the setup process of the \hyperlink{abbr:ci}{CI} pipeline can be simplified to reduce the amount of boilerplate code.
% Differet approaches
Therefore, different \textit{Auto-Scaler} and \hyperlink{abbr:ci}{CI} pipeline implementation approaches for further research are introduced.



\subsection{Proactive Auto-Scaler Approach}
% Currently
The current implementation of the \textit{Auto-Scaler} follows a reactive approach. It uses static threshold-based rules (e.g., maximum and minimum \hyperlink{abbr:cpu}{CPU} utilization) to define when a scaling action is necessary.
% Not sufficient
The \textit{Auto-Scaler} can be optimized using a proactive approach. A proactive \textit{Auto-Scaler} tries to predict the computing environment's future state (e.g., reinforcement learning) using historical data.
% Extansion of the MAPE architecture
The \hyperlink{abbr:mape}{MAPE} architecture can be extended with a knowledge source (MAPE-K), which provides data about the computing environment \cite{Sinreich2006AnAB}. The autonomic manager can use the knowledge to try to predict when a scaling action is necessary.


\subsection{Vertical-Scaling Approach}
% evaluations
The evaluation of the static \hyperlink{abbr:cpu}{CPU}-only worker experiments showed that from a specific number of workers, no significant performance improvement was made.
% vertical
Therefore, instead of scaling the replicas of worker nodes, a vertical-scaling approach, where the \textit{Auto-Scaler} scales the available computing resources of a worker node, should be explored.
% results
Providing more powerful resources, increases the computational power of the Apache Spark cluster and could lead to lower execution times for training machine learning models.


\subsection{Scaling GPU-accelerated Workers}
\label{subsec:08_outlook_gpus}
% Currently no GPUs
Due to the lack of implementation, the \textit{Auto-Scaler} can not keep track of available \hyperlink{abbr:gpu}{GPUs} in the cluster. Therefore, it was not possible to evaluate the \textit{Auto-Scalers} impact on \hyperlink{abbr:gpu}{GPU}-accelerated worker nodes.
% GPU approach
Extending the \textit{Auto-Scaler} to keep track of \hyperlink{abbr:gpu}{GPUs} which are free or already allocated by a worker should be explored.
%
Auto-scaling \hyperlink{abbr:gpu}{GPU}-accelerated worker nodes can improve the spark-job execution time.


\subsection{Save Temporary Data on an External Service}
% The problem
With the current \textit{Auto-Scaler} implementation, it is not possible to remove worker nodes while an application is actively performing because worker nodes save important temporary data.
% A solution
The worker nodes can be extended to save temporary data on an external storage service, which enables the \textit{Auto-Scaler} to remove nodes while an application performs.
% 
The impact of removing nodes during the execution of an application should be explored. In the evaluation of this thesis, only the impact of adding worker nodes is explored.


\subsection{Train-Stage Docker Image}
% Intro
GitLab CI/CD requires to setup a pipeline for each project additionally using a \texttt{.gitlab-ci.yml} configuration file.
% Current approach
Therefore, the developer has to implement the process to share the source code between multiple Docker containers everytime a new project is created. Additionally, the custom submit script needs to be added to the project source code.
% simplification
This process can be automated using a custom Docker image, which is then used for the \textit{train-stage} implementation. The image automatically copies the source code to a \textit{mount-point} and integrates the custom submit script.
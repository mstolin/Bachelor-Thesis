\chapter{Evaluation}
\label{chap:07_evaluation}

In this chapter, the implementation of \Chap{chap:06_implementation} is evaluated in various experiments. Furthermore, the experimental results are discussed.

\section{Experimental Environment}
%
The experiments are conducted on the DGX introduced in \Sec{sec:06_env}.
% 2 GPUs
As for the implementation, 2 GPUs have been available for the experiments.


% ===========================================
% ===========================================
\section{Experiments}
% Intro
The performance of the computing environment is measured on two widely used machine learning algorithms.
% Explain
In both experiments, a machine learning model is trained on the computing environment using Apache Spark. Furthermore, each benchmark is evaluated using three different configurations:
\begin{enumerate}
\item Using a static number of CPU-only Apache Spark workers
\item Using a static number of GPU-accelerated Apache Spark workers
\item Dynamically scaling the replicas of CPU-only Apache Spark workers using the \textit{Auto-Scaler}
\end{enumerate}
% 10 times
For each benchmark configuration, the experiment is conducted 10 times. The spark-job execution-time mean value over all 10 iterations per experiment is used as result.
% DIfference between GPU and CPU
The performance difference between GPU accelerated worker nodes and CPU-only worker nodes is explored using a CPU-only, and a GPU-only version of the benchmark implementation.

\paragraph{}
% Why no auto gpu
The \textit{Auto-Scaler} is not evaluated using GPU-accelerated worker nodes. As mentioned, for this experiment only 2 GPUs are available. The RAPIDS plugin allocates a GPU exclusively for each executor. When the \textit{Auto-Scaler} creates a new worker node, the worker allocates a random available GPU.
% Live system
However, the host machine is a live system and it is not possible to allocate a random GPU which may be in use by another application. Furthermore, the \textit{Auto-Scaler} does not support to allocate a specific GPU for each newly created worker. An approach to overcome this problem is introduced in \Sec{subsec:08_outlook_gpus}.

\paragraph{}
% The algorithm
The two benchmarks to evaluate the performance are:
\begin{itemize}
\item XGBoost classification model using the \textit{Fannie Mae’s Single-Family Historical Loan Performance Dataset}\footnote{Downloaded from: \url{https://docs.rapids.ai/datasets/mortgage-data} (Accessed: 2021-02-06)}\cite{Fannie2021Mortgage}

\item XGBoost regression model using a Taxi fare dataset\footnote{The Taxi dataset is available at: \url{https://github.com/NVIDIA/spark-xgboost-examples/tree/spark-3} (Accessed: 2021-02-06)}
\end{itemize}
% Source
The source code and the dataset used in these experiments are available on Github on the \textit{spark-xgboost-examples}\footnote{spark-xgboost-examples - \url{https://github.com/NVIDIA/spark-xgboost-examples/tree/spark-3} (Accessed: 2021-02-06)} repository from NVIDIA.
% The two diff impl conf
The repository provides a \textit{mortgage} and a \textit{taxi} application. Both applications provide a CPU and a GPU configuration.

% supervised
Classification and regression algorithms are supervised machine learning algorithms.
% whats supervised
The goal of supervised machine learning algorithms is to train a model by finding patterns in labelled data. Then, the model is used to predict labels on new data based on the learned labels.
% Classification
The classification algorithm identifies the category of a label.
% Regression
A regression algorithm predicts a continuous numeric value \cite{Mcdonald2020SparkRapids}.


% ===========================================
% ===========================================
\section{Static Worker CPU only}
\label{sec:07_static}
% Intro
The first experiment is conducted using a fixed number of CPU-only Apache Spark workers to evaluate the performance of both benchmarks.
% Appendix
The performance for each iteration is available at \Section{sec:appendix_eval}.
% results
\begin{figure}[h]
\centering
\includegraphics[scale=0.9]{images/07_evaluation/overall_cpu}
\caption{Mean execution time using CPU-only worker nodes}
\label{fig:07_static_results}
\end{figure}
% Explain fig
\Fig{fig:07_static_results} illustrates the mean execution time for both benchmarks.

\paragraph{}
% best results
The best performance for the classification benchmark was achieved with 247 seconds using 30 workers with an improvement of 7.04x in comparison to 1 worker. The regression benchmark achieved best with 102 seconds using 10 workers. The execution has an improvement of 3.21x in comparison to 1 worker. In addition, the regression benchmark achieved an execution time of 102 seconds by using 30 workers as well. However, using 10 workers is considered the best result for the regression benchmark because less workers have been used.

\paragraph{}
% The improvement curve
For the classification benchmark, the execution time had no significant improvement from 15 workers. Using 20 workers increased the execution time by 18 seconds. Using 35 workers improved the execution time by only 5 seconds in comparison to 15 workers.
% regression
The regression benchmark shows a similar result. As mentioned, the best result was achieved with 10 workers. The same results was achieved with 30 workers. Furthermore, using 35 workers increased the best performance by 7 seconds.


% and?
This result shows that over-provisioning the Apache Spark cluster by adding more workers from a certain number of workers can have a negative impact on the performance.


% ===========================================
% ===========================================
\section{GPU Acceleration}
% Intro
The impact of GPU-accelerated Apache Spark worker nodes for both benchmarks have been evaluated using 2 configurations: Using 1 GPU accelerated worker node, and using 2 GPU accelerated worker nodes.
%
The performance for each GPU experiment is available at ANHANG A.
% vs figure
\begin{figure}[h]
\centering
\includegraphics[scale=0.9]{images/07_evaluation/overall_cpu_vs_gpu}
\caption{Mean execution time of GPU-accelerated worker nodes vs. Mean execution time of CPU-only worker nodes}
\label{fig:07_gpu_results}
\end{figure}
% Explain results
The results of the experiment are illustrated in \Fig{fig:07_gpu_results}.
% vs
The mean execution time using GPU accelerated worker nodes are plotted against the mean execution time using CPU-only worker nodes.
% overall
Overall, GPU accelerated worker nodes significantly outperformed the equivalent number of CPU-only worker nodes.
% classification
For the classification benchmark, the best execution-time was achieved using 2 GPU-accelerated worker nodes with 342 seconds in comparison to 828 seconds using 1 CPU-only worker node. This an 2.42x improvement. Using 1 GPU-accelerated worker node achieved an improvement of 2.91x with 597 seconds in comparison to 1 CPU-only worker node with 1740 seconds.
% regression
The regression benchmark achieved best with 2 GPU-accelerated worker nodes needing 342 seconds in comparison of 2 CPU-only worker nodes needing 828 seconds. Therefore, using 2 GPU-accelerated worker nodes achieved an improvement of 3.02x. Using 1 GPU-accelerated worker node achieved an improvement of 4.55x in comparison to 1 CPU-only worker node.

\paragraph{}
% The table
\Tab{table:07_gpu_overall_results} shows the overall results by comparing the CPU and GPU results for each benchmark.
% GPU outperforms
Overall, GPU-accelerated worker nodes were able to outperform their CPU-only equivalent.
% classification
However, for the classification benchmark, 5 CPU-only worker nodes achieved an improvement of 62 seconds compared to 2 GPU-accelerated worker nodes. In addition, 10 CPU-only worker nodes outperformed 2 GPU-accelerated worker nodes by 47 seconds.
% regression
Otherwise, for the regression benchmark no CPU-only configuration was able to improve the performance of GPU-accelerated worker nodes.


% Results as table
\begin{table}[ht]
\centering
\begin{tabular}{@{}l|ll|ll@{}}
\toprule
                  & \multicolumn{2}{c|}{Classification}                & \multicolumn{2}{c}{Regression}                    \\
Number of workers & \multicolumn{1}{c}{CPU} & \multicolumn{1}{c|}{GPU} & \multicolumn{1}{c}{CPU} & \multicolumn{1}{c}{GPU} \\ \midrule
1  & 1740s & 597s & 328s & 72s \\
2  & 828s  & \textbf{342s} & 139s & \textbf{46s} \\
5  & 535s  & -      & 129s & -     \\
10 & 295s  & -      & \textbf{102s} & -     \\
15 & 263s  & -      & 109s & -     \\
20 & 281s  & -      & 120s & -     \\
25 & 257s  & -      & 129s & -     \\
30 & \textbf{247s}  & -      & 102s & -     \\
35 & 258s  & -      & 109s & -     \\ \bottomrule
\end{tabular}
\caption{Mean execution of all CPU-only and GPU-accelerated experiments for both benchmarks}
\label{table:07_gpu_overall_results}
\end{table}


% ===========================================
% ===========================================
\section{Auto-Scaler}
% Intro
To test the impact of the \textit{Auto-Scaler} while training machine learning applications, both benchmarks have been tested with different \textit{Auto-Scaler} configurations. 


\subsection{Auto-Scaler Benchmark Configurations}
% Wheres it from
The configuration parameters are chosen in accordance to the results of the static worker experiment (\Sec{sec:07_static}).
% Table
The \textit{Auto-Scaler} configuration parameters for each benchmark are shown in \Tab{table:07_auto-scaler_config_parameter}.
% Iteration Figure
\begin{figure}[h]
\centering
\includegraphics[scale=0.9]{images/07_evaluation/overall_auto-scaler_iterations}
\caption{CPU utilization during all iterations of both benchmarks with 2 CPU-only workers}
\label{fig:07_auto-scaler_iterations_results}
\end{figure}
% Max worker
To prevent an cluster over-provisioning, the maximum number of worker nodes for the classification benchmark is set to 15, and 10 for the regression benchmark.
% recurrence
\Fig{fig:07_auto-scaler_iterations_results} shows all 10 iterations of the CPU-only experiment for both benchmarks. It shows that, while the machine learning model is trained, only 2 performance spikes occurred at the beginning and at the end during a short time period. To scale while these performance spikes occur, the recurrence factor is set to 1. Additionally, the maximum CPU utilization for both benchmarks is set in accordance to the maximum CPU utilization during the performance spikes.
% The configurations
\begin{table}[ht]
\centering
\begin{tabular}{@{}l|ll@{}}
\toprule
Parameter               & Classification & Regression \\ \midrule
Interval                & 5 seconds      & 5 seconds  \\
Recurrence factor       & 1              & 1          \\
Cooldown period         & 60 seconds     & 60 seconds \\
Target CPU utilization  & 10\%           & 5\%        \\
Minimum CPU utilization & 5\%           & 2\%       \\
Maximum CPU utilization & 25\%           & 10\%       \\
Minimum worker nodes    & 2              & 2         \\
Maximum worker nodes    & 15              & 10         \\ \bottomrule
\end{tabular}
\caption{\textit{Auto-Scaler} configuration parameters for both benchmarks}
\label{table:07_auto-scaler_config_parameter}
\end{table}


\subsection{Execution Time Evaluation}
% Figure
\begin{figure}[h]
\centering
\includegraphics[scale=0.9]{images/07_evaluation/overall_auto-scaler}
\caption{\textit{Auto-Scaler} experiment mean execution time of both benchmarks}
\label{fig:07_auto-scaler_results}
\end{figure}

\paragraph{}
% Figure
\Fig{fig:07_auto-scaler_results} illustrates the results of the \textit{Auto-Scaler} experiment. The mean execution time of 2 CPU-only worker nodes is plotted against the mean execution time of the \textit{Auto-Scaler} experiment.
% Explain
The results show, that no improvement was achieved using the \textit{Auto-Scaler} implementation.
% Results
For the classification benchmark, the execution time of 2 CPU-only worker nodes increased by 228 seconds when enabling the \textit{Auto-Scaler} in the environment. Furthermore, the regression benchmark execution for 2 CPU-only worker nodes increased by 38 seconds with the the \textit{Auto-Scaler} enabled.

\paragraph{}
% Table
\begin{table}[ht]
\centering
\begin{tabular}{@{}l|ll|ll@{}}
\toprule
                               & \multicolumn{2}{c|}{Classification}                   & \multicolumn{2}{c}{Regression}                       \\
\multicolumn{1}{c|}{Iteration} & \multicolumn{1}{c}{Time} & \multicolumn{1}{c|}{Nodes} & \multicolumn{1}{c}{Time} & \multicolumn{1}{c}{Nodes} \\ \midrule
1  & \textbf{840s} & 15 & 168s          & 5 \\
2  & 960s          & 7  & \textbf{150s} & 2 \\
3  & 1020s         & 15 & 186s          & 6 \\
4  & 1140s         & 15 & 180s          & 6 \\
5  & 1080s         & 15 & 180s          & 6 \\
6  & 1140s         & 15 & 180s          & 6 \\
7  & 1140s         & 15 & 180s          & 6 \\
8  & 1080s         & 15 & 180s          & 6 \\
9  & 1140s         & 7  & 180s          & 6 \\
10 & 1020s         & 15 & 186s          & 6 \\ \bottomrule
\end{tabular}
\caption{Results of each iteration during the \textit{Auto-Scaler} experiment for both benchmarks}
\label{table:07_auto-scaler_iterations}
\end{table}
% Explain
\Tab{table:07_auto-scaler_iterations} summarizes the results for all 10 iterations of each benchmark.
% classification
For the classification benchmark, the best result was achieved in the first iteration with an execution time of 960 seconds and a maximum number of 15 worker nodes during the execution.
% regression
The regression benchmark achieved best in the second iteration, with an execution time of 150 seconds and a maximum number of 2 worker nodes during this iteration.
% n maximum reached
The results show that during the classification benchmark experiment, the \textit{Auto-Scaler} added the maximum number of allowed worker nodes in 8 out of 10 iterations. During the regression benchmark experiment, the maximum number of worker nodes was never reached. Furthermore, in the second iteration, when the best result was achieved, the \textit{Auto-Scaler} has not scaled the number of worker nodes.

\paragraph{}
% Meaning
The performance decrease when enabling the Auto-Scaler is possibly caused by the lack of efficient workload distribution by Apache Spark.


\section{Experimental Results Summary}
% Intro
The experimental results show, that GPU-accelerated worker nodes significantly outperform worker nodes using only CPUs.
% Auto-scaler
However, the \textit{Auto-Scaler} was not able to improve the performance of the computing environment.
%
Moreover, enabling the \textit{Auto-Scaler} during the training of machine learning models increased the execution time of active performing spark-jobs.

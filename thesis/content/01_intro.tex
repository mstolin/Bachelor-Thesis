\chapter{Introduction}
\label{sec:introduction}
%
\todo{EVTL DATUM AUF NOVEMBER ÄNDERN; WEGEN STARTDATUM!! in den online quellen}

%Complex computation operations rely on a IT infrastructure that is able to perform operations on scale. Computing systems that automatically adapt to demands and conditions of the workload serve the needs of
%highly scaled computing systems QUELLE. The microservice architecture is a technique to divide a big application into independent services. Each single service only serves one business task of the application QUELLE. In recent years, container applications have become popular to deploy microservices and visualize the needed resources for performing complex operations QUELLE. A container is a single unit that serves the 
%whole environment for a microservice.

% -- Struktur von Intro
% Grid computing
% Autonomic computing
% Microservices ??
% Container
% GPU acceleration


% Start
Storing huge amounts of data has become inexpensive in recent years, but processing it, requires 
parallel computations on clusters with multiple machines \cite{Chambers2018Spark}. Complex 
computation operations rely on a IT infrastructure with the ability to perform operations on scale. 
%Scalability is the ability of a system to grow in accordance of the amount of work \cite{TakadaDistr}.

% Sachen aus autonomic computing
In order to achieve high scalability, computing systems need to adapt dynamically to demands and conditions of the workload.


% Technologies
%In recent years, technologies have contributed to improve the scalability and performance of modern 
%computing systems like cloud computing, virtualization and GPU acceleration.

% Autonomic computing
%Autonomic computing is an architecture for computing systems to manage themselves in accordance to high level objectives, %configured by administrators. These computing systems dynamically adapt to demands and conditions of the workload. The f%our main attributes of autonomic computing are self-configuring, self-healing, self-optimizing and self-protecting \%cite{Kephart2003VisionComputing}. The autonomic computing architecture consists 
%of an autonomic manager, which is responsible to make adjustments in it`s own environment \cite{Sinreich2006AnAB}. %Technologies like Docker help to create containers ... bla bla bla

% Virtualization
%Modern virtualization technologies like containers help to overcome scalability limitations with the creation of virtual instances %of IT resources \cite{Erl2013CloudComputing}.


% GPU acceleration
% In addition to elasticity
%Another aspect to improve complex computation operations is GPU acceleration. GPU accelerations has been an on-going %research area in the recent years.

\section{Problem statement}

%% Struktur
% Möglichkeit IT resourcen automatisch nach bedarf zu skalieren
% Viele ETL operation sind daten intensiv von natur, GPU kann helfen

%To prevent a breakdown of the IT environment while complex operations are being computed, the IT environment should have the ability to scale IT resources dynamically
%when performance peaks and fluctuations occur.

% Dynamisch auf performance spitzen reagieren
ETL \footnote{Extract transform load} operations are compute-intensive. During the execution of analytic applications, performance thresholds can be reached and
the computing system can become out-of-order. 

In addition, many algorithms profit from data-parallelism. % Das ha mehr mit SPark zu tun

%If performance thresholds are reached and the IT environment can`t adapt 
%dynamically to the usage demand, the environment can look temporarily out-of-order. 
%During the execution of operations, performance threshold can be reached and leads to 

% Mit GPU ETLs beschleunigen
%In addition, algorithms do profit from data-parallelism. 

\section{Research questions}

\section{Thesis structure}
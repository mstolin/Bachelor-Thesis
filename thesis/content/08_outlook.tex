\chapter{Outlook}
\label{chap:08_outlook}

This chapter discusses the results of the evaluation. Furthermore, an outlook is given for further research.


% ===========================================
% ===========================================
\section{Results}


\subsection{GPU accelerated Apache Spark Worker}
% The results
\begin{table}[]
\centering
\begin{tabular}{@{}l|ll|ll@{}}
\toprule
  & \multicolumn{2}{c|}{Classification} & \multicolumn{2}{c}{Regression} \\
\multicolumn{1}{c|}{\# Workers} & \multicolumn{1}{c}{CPU} & \multicolumn{1}{c|}{GPU} & \multicolumn{1}{c}{CPU} & \multicolumn{1}{c}{CPU} \\ \midrule
1 & 1740s       & 597s        & 328s      & 72s     \\
2 & 828s       & 342s        & 139s      & 46s     \\ \bottomrule
\end{tabular}
\caption{Evaluation results of CPU vs. GPU}
\label{table:08_results_cpu-gpu}
\end{table}
% describe results
An overview about the results of the impact of GPU accelerated  Apache Spark workers is provided in \Tab{table:08_results_cpu-gpu}.


% classification
For the classification benchmark, the execution time had an improvement of 2.9x by using 1 GPU accelerated worker node. Using 2 GPU accelerated nodes impoved the execution by 2.4x.
% regression
For the regression benchmark, an improvement of 4.5x was achieved by using 1 GPU and 3x for 2 GPUs.


\subsection{Auto-Scaling}
Given the result of TABLE XY, it is clear, that the Auto-Scaler had an overall bad impact on the execution time.
% Classification

% Regression
For the regression benchmark, the Auto-Scaler experiment increased the mean execution time by 38 seconds.


% ===========================================
% ===========================================
\section{Outlook}


\subsection{Scaling}
Current approach: Wait until all applications have finished.
Better approach: Blacklist worker for removing so no executor will be launchend on the worker.
Create an external shuffle service, so worker can be remove on runtime.


\subsection{Reinforcement Learning for Auto-Scaling}
bla bla


\subsection{Pro Active Auto-Scaler}
Nicht ganz sicher ob Pro Active hier das richtige Wort ist
Punkt is: KHPA ist kacke frür Spark -> Reinforcement Learning besser.
Noch besser: Gew. Anzahl an Worker bereit stellen und erst bei bedarf aktivieren. Docker up Zeit sparen

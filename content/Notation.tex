\chapter*{Notation}

\section*{Konventionen}
%
\begin{tabular}{rl}
$x$ & Skalar
\\
$\vx$ & Spaltenvektor
\\
$\x, \rvx$ & Zufallsvariable/-vektor
\\
$\hx, \hvx$ & Mittelwert/-vektor
\\
$x^*, \vx^*$ & Optimaler Wert/Vektor
\\
$x_{0:k}, \vx_{0:k}$ & Folge von Werten $(x_0, x_1, \ldots, x_k)$ / Vektoren $(\vx_0, \vx_1, \ldots, \vx_k)$
\\
$\A$ & Matrizen
\\
$\SA$ & Mengen
\\
$\preceq, \prec$ & schwache/strenge Pr�ferenzrelation
\\
$\NewR$ & Reelle Zahlen
\\
$\NewN$ & Nat�rliche Zahlen
\\
$\blacksquare$ & Ende eines Beispiels
\\
$\square$	& Ende eines Beweises
\end{tabular}

\section*{Operatoren}
%
\begin{tabular}{rl}
$\mat A\T$ & Matrixtransposition
\\
$\mat A^{-1}$ & Matrixinversion
\\
$|\mat A|$ & Determinante einer Matrix
\\
$|\SA|$ & Kardinalit�t der Menge $\SA$
\\
$\pot(\SA)$ & Potenzmenge von $\SA$
\\
$\E\{\cdot\}$ & Erwartungswertoperator
\\
$\SO(g)$ & O-Kalk�l entsprechend der Landau-Notation bei welcher \\&beispielsweise $f(x) \in \SO(g(x))$ besagt, dass die Funktion \\&$f(x)$ die Komplexit�t $\SO(g(x))$ besitzt
\end{tabular}

\section*{Spezielle Funktionen}
%
\begin{longtable}{rl}
$\Pr(\SE)$ & Wahrscheinlichkeitsma�, welches die Wahrscheinlichkeit
\\ & angibt, dass Ereignis $\SE$ eintritt
\\
$p(\vx)$ & (Wahrscheinlichkeits-)Dichtefunktion f�r kontinuierliche $\vx$ \\
& und Z�hldichte f�r diskrete $\vx$
\\
$p(\vx|\vy)$ & Bedingte Dichtefunktion
\\
$P(\vx)$ & (Wahrscheinlichkeits-)Verteilungsfunktion
\\
$\erf(x)$ & Gau�'sche Fehlerfunktion
\\
$\exp(x)$ & Exponentialfunktion $e^x$
\\
$\Gauss(\vx; \hvx, \C_x)$ & Gau�dichte, \dh Dichtefunktion eines normalverteilten \\&Zufallsvektors $\rvx$ mit Mittelwertvektor $\hvx$ und\\& Kovarianzmatrix $\C_x$
\end{longtable}
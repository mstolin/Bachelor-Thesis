\chapter{Background}
\label{sec:demo}
%
In this chapter bla bla

\section{Autonomic computing}
Autonomic computing is an architecture for computing systems to enable the ability to manage themselves in accordance to high level objectives configured by administrators \cite{Kephart2003VisionComputing}. 
These computing systems dynamically adapt to demands and conditions of the workload \cite{Kephart2003VisionComputing}.
The four main attributes of autonomic computing are self-configuring, self-healing, self-optimizing and self-protecting  \
. The autonomic computing architecture consists of an autonomic manager and managed resources.
% Autonomic manager, managed resources and knowledge source <- the architecture

\subsection{Autonomic manager}
The autonomic manager implements an intelligent control-loop to collect system metrics and acts according to the collected details. It can only make adjustments within it`s own scope and uses policies to make decisions of what actions have to be
executed to accommodate the objectives.
To be self-managing, the autonomic manager has to implement the following four automated functions.

\begin{enumerate}
\item \textbf{Monitor}
- The monitor function is responsible to collect the needed metrics from all managed resources and applies aggregation and filter
operations to the collected data. After that the function reports the metrics.

\item \textbf{Analyze}
- To determine if changes have to be made to the computing system, the collected data has to be analyzed.

\item \textbf{Plan}
- If changes have to be made, an appropriate change plan has to be generated. A change plan consists of actions that are needed to achieve the configured goals and objectives. The change plan needs to be forwarded to the execute function.

\item \textbf{Execute}
- The execute function applies all necessary changes to the computing system.

\end{enumerate}

Multiple autonomic manager can exist in an autonomic computing environment to perform only certain parts. For example, 
there can be one autonomic manager which is responsible to monitor and analyze the system and another autonomic manager 
to plan and execute. To create a complete and closed control-loop multiple autonomic manager can be composed together.

\section{Docker}


\section{Apache Spark}


\section{Prometheus}


\section{NVIDIA RAPIDS}


\section{Gitlab CI/CD}
